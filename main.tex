% main.tex

% Use the ACM large 1-column format class from your template directory
\documentclass[acmlarge]{template/column-format-template/acmart}

% Set up any additional package imports here
\usepackage{graphicx}     % For including graphics
\usepackage{amsmath}      % For advanced math formatting
\usepackage{algorithm}    % For algorithm pseudocode
\usepackage{algorithmic}  % For algorithm pseudocode
\usepackage{booktabs}     % For professional tables
\usepackage{listings}     % For code listings
\usepackage{subcaption}   % For subfigures
\usepackage{url}          % For URLs in bibliography
\usepackage{titlesec}     % For custom formatting
\usepackage{xcolor}       % For colors
\usepackage{listings}     % For code snippets
\usepackage{subcaption}   % For subfigures and subtables
\usepackage{hyperref}     % For links and references

% Custom settings for listings
\lstset{
    language=Python,                    % Sets the programming language for syntax highlighting
    basicstyle=\footnotesize\ttfamily,  % Sets the font style and size for the code
    keywordstyle=\color{blue},          % Style applied to keywords (e.g., "def", "import")
    commentstyle=\color{gray},          % Style applied to comments (e.g., lines starting with "#")
    stringstyle=\color{red},            % Style applied to strings (e.g., text within quotes)
    % backgroundcolor=\color{gray!10},  % Background color for the code block (light gray in this case)
    % numbers=left,                     % Displays line numbers on the left side of the code block
    % numberstyle=\tiny,                % Font size/style for line numbers
    % stepnumber=1,                     % Line number increment (e.g., every 1 line gets a number)
    % numbersep=5pt,                    % Distance between the line numbers and the code
    linewidth=0.95\linewidth,           % Code block occupies 80% of text width
    xleftmargin=-20pt,                  % Moves the listing 20pt closer to the left margin
    xrightmargin=5pt,                   % Optional: Adds extra padding on the right
    breaklines=true,                    % Automatically breaks long lines to fit within the page width
    captionpos=b,                       % Places the caption below the code block
    abovecaptionskip=5pt,               % Adjusts the space above the caption to 5pt
    belowcaptionskip=8pt,               % Adjusts the space below the caption to 8pt
}

% Custom settings for hyperref
% FIXME: update and fix
\hypersetup{
    colorlinks=false,            % Disable colored links (use borders instead)
    linkbordercolor=1 0 0,       % Red border around internal links
    citebordercolor=0 1 0,       % Green border around citations
    urlbordercolor=0 1 1,        % Cyan border around URLs
    filebordercolor=0 .5 .5,     % Teal border around file links
    pdfpagemode=UseOutlines,     % Open the PDF with bookmarks visible
}

% Remove ACM-specific references and permissions
\setcopyright{none} % Disable copyright
\makeatletter
\@printpermissionfalse
\@printcopyrightfalse
\@acmownedfalse
\makeatother

% Remove ACM reference format
\settopmatter{printacmref=false}
\renewcommand\footnotetextcopyrightpermission[1]{}

% Optional: Plain page style
\pagestyle{plain}

% TODO: check
% \BibTeX command to typeset BibTeX logo in the docs
\AtBeginDocument{%
  \providecommand\BibTeX{{%
    Bib\TeX}}}

% TODO: check
% Define paths for graphics and bibliography
% \graphicspath{{template/column-format-template/}}
% \bibliographystyle{template/column-format-template/ACM-Reference-Format}

% TODO: check
% \graphicspath{{figures/}}                 % Single directory
% \graphicspath{{fig1/}{fig2/}}             % Multiple directories

% TODO: check
% \bibliographystyle{ACM-Reference-Format}  % Standard ACM format
% \bibliographystyle{plain}                 % Basic format
% \bibliographystyle{ieeetr}                % IEEE format

% Optional: Customize section titles
% \titleformat{\section}
%   {\LARGE}                      % Large, not bold
%   {\thesection}                 % Section number
%   {0.5em}                       % Space between number and title
%   {}                            % Code before title body

% Optional: Customize subsection titles
% \titleformat{\subsection}
%   {\Large}                      % Large (but smaller than section), not bold
%   {\thesubsection}              % Subsection number
%   {0.5em}                       % Space between number and title
%   {}                            % Code before title body

% Optional: Adjust spacing before and after sections
% \titlespacing*{\section}{0pt}{2.5ex plus 1ex minus .2ex}{1.5ex plus .2ex}
% \titlespacing*{\subsection}{0pt}{2.25ex plus 1ex minus .2ex}{1.5ex plus .2ex}

\begin{document}

% Title
\title{SSH Shell Attacks}

% Define authors and their affiliations

\author{Andrea Botticella}
\authornote{The authors collaborated closely in developing this project.}
\email{andrea.botticella@studenti.polito.it}
\affiliation{%
  \institution{Politecnico di Torino}
  \city{Turin}
  \country{Italy}
}

\author{Elia Innocenti}
\authornotemark[1]
\email{elia.innocenti@studenti.polito.it}
\affiliation{%
  \institution{Politecnico di Torino}
  \city{Turin}
  \country{Italy}
}

\author{Renato Mignone}
\authornotemark[1]
\email{renato.mignone@studenti.polito.it}
\affiliation{%
  \institution{Politecnico di Torino}
  \city{Turin}
  \country{Italy}
}

\author{Simone Romano}
\authornotemark[1]
\email{simone.romano@studenti.polito.it}
\affiliation{%
  \institution{Politecnico di Torino}
  \city{Turin}
  \country{Italy}
}

% Short author list for page headers
\renewcommand{\shortauthors}{Botticella, Innocenti, Mignone, and Romano}

% CCS Concepts
\begin{CCSXML}
<ccs2012>
   <concept>
       <concept_id>10010147.10010257.10010321.10010337</concept_id>
       <concept_desc>Computing methodologies~Supervised learning by classification</concept_desc>
       <concept_significance>500</concept_significance>
   </concept>
   <concept>
       <concept_id>10010147.10010257.10010321.10010335</concept_id>
       <concept_desc>Computing methodologies~Unsupervised learning</concept_desc>
       <concept_significance>500</concept_significance>
   </concept>
   <concept>
       <concept_id>10010147.10010257.10010339</concept_id>
       <concept_desc>Computing methodologies~Natural language processing</concept_desc>
       <concept_significance>400</concept_significance>
   </concept>
   <concept>
       <concept_id>10010147.10010257.10010236</concept_id>
       <concept_desc>Computing methodologies~Machine learning</concept_desc>
       <concept_significance>400</concept_significance>
   </concept>
   <concept>
       <concept_id>10010147.10010528.10010531</concept_id>
       <concept_desc>Computing methodologies~Machine learning approaches</concept_desc>
       <concept_significance>300</concept_significance>
   </concept>
   <concept>
       <concept_id>10002978.10003006.10011634</concept_id>
       <concept_desc>Security and privacy~Intrusion detection systems</concept_desc>
       <concept_significance>200</concept_significance>
   </concept>
</ccs2012>
\end{CCSXML}

\ccsdesc[500]{Computing methodologies~Supervised learning by classification}
\ccsdesc[500]{Computing methodologies~Unsupervised learning}
\ccsdesc[400]{Computing methodologies~Natural language processing}
\ccsdesc[400]{Computing methodologies~Machine learning}
\ccsdesc[300]{Computing methodologies~Machine learning approaches}
\ccsdesc[200]{Security and privacy~Intrusion detection systems}

% Keywords
\keywords{Machine learning, supervised learning, unsupervised learning, language models, text classification, clustering, intent classification, SSH shell attacks, security log analysis}

% 00-abstract.tex

% Abstract: A concise summary of the project, including the main objectives, methods used, and key findings.

\begin{abstract}

% Guidelines for abstract:
% - Usually 150-250 words
% - Can include formatting:

This paper presents \textit{novel approach} to...
We achieve an improvement of \textbf{20\%} over...

Our main contributions are:

\begin{itemize}
    \item First contribution
    \item Second contribution
\end{itemize}

\end{abstract}


\maketitle

% Table of Contents
\setcounter{tocdepth}{1}
\tableofcontents

% TODO: remove some space here (?)

% Include the sections you have in your sections directory
% Section Title
\section{Introduction} % Replace with the appropriate title, e.g., Methodology, Results, etc.

% Main Content
This section introduces the topic of the project, provides background information, and outlines the objectives.

% Subsections (if needed)
\subsection{Motivation}
Provide an explanation of why this topic is important and relevant.

\subsection{Objective}
Clearly state the objectives and what the project aims to accomplish.


% 02-background.tex

% Background: Explanation of the context, including the importance of security logs, the MITRE ATT&CK framework, and the dataset used.

% Section Title
\section{BACKGROUND}

    % Main Content

    \subsection{Security Logs and Attack Analysis}
    
        Security logs represent a critical source of information for understanding system vulnerabilities and potential cyber attacks. These logs capture detailed records of system events, network interactions, and user activities, providing valuable insights into potential security breaches.

        In the context of SSH shell attacks, logs document the sequence of commands executed during a malicious session, enabling security researchers to analyze attacker behaviors, techniques, and potential system impacts. However, the manual analysis of these logs is challenging due to their volume, complexity, and often non-standard formatting.

    \subsection{MITRE ATT\&CK Framework}
    
        The MITRE ATT\&CK (Adversarial Tactics, Techniques, and Common Knowledge) framework provides a comprehensive knowledge base of adversary tactics and techniques observed in real-world cyber attacks. This framework serves as a standardized methodology for understanding and categorizing attack strategies.

        \noindent For our research, we focus on seven key intents derived from the MITRE ATT\&CK framework:

        \begin{itemize}
            \item \textbf{Persistence:} Techniques used by adversaries to maintain system access across restarts or credential changes.
            \item \textbf{Discovery:} Methods for gathering information about the target system and network environment.
            \item \textbf{Defense Evasion:} Strategies to avoid detection by security mechanisms.
            \item \textbf{Execution:} Techniques for running malicious code on target systems.
            \item \textbf{Impact:} Actions aimed at manipulating, interrupting, or destroying systems and data.
            \item \textbf{Other:} Less common tactics including Reconnaissance, Resource Development, Initial Access, etc.
            \item \textbf{Harmless:} Non-malicious code or actions.
        \end{itemize}

    \subsection{Dataset Overview}
    
        Our research utilizes a comprehensive dataset of SSH shell attacks collected from a honeypot deployment. Key characteristics of the dataset include:

        \begin{itemize}
            \item Approximately 230,000 unique Unix shell attack sessions
            \item Recorded after SSH login
            \item Stored in Parquet format for efficient data processing
            \item Columns include:
            \begin{itemize}
                \item Session ID
                \item Full session text (command sequences)
                \item Timestamp
                \item Intent labels based on MITRE ATT\&CK tactics
            \end{itemize}
        \end{itemize}

        It is important to note that the dataset's labels are automatically generated through a research project and may contain potential classification errors. This inherent uncertainty adds an additional layer of complexity to our analysis and highlights the importance of robust machine learning techniques.

    \subsection{Research Approach}
    
        Our research employs a multi-faceted approach to SSH shell attack log analysis:

        \begin{itemize}
            \item Comprehensive data exploration and preprocessing
            \item Supervised learning for attack intent classification
            \item Unsupervised learning for attack pattern discovery
            \item Advanced language model exploration
        \end{itemize}

        By combining these techniques, we aim to develop a comprehensive framework for understanding and categorizing SSH shell attacks, ultimately contributing to improved cybersecurity threat detection and response strategies.

% 03-data-exploration-and-pre-processing.tex

% Data Exploration and Pre-processing
% 3.1. Introduction: Introduces the data exploration and pre-processing tasks.
% 3.2. Dataset Preparation: Describes the process of loading the dataset and initial inspection.
% 3.3. Temporal Analysis: Analyzes when the attacks were performed.
% 3.4. Feature Extraction: Extracts features from the attack sessions.
% 3.5. Common Words Analysis: Identifies the most common words in the sessions.
% 3.6. Intent Distribution: Analyzes the distribution of intents over time.
% 3.7. Text Representation: Converts text into numerical representations (BoW, TF-IDF).

% Section Title
\section{DATA EXPLORATION AND PRE-PROCESSING}

    % Main Content

    \subsection{Introduction}
        
        This section outlines the steps taken to explore and preprocess the dataset used in this study. The primary objective is to understand the data characteristics, identify patterns, and prepare the data for further analysis. We focus on temporal trends, intent distributions, and textual features.
        
    \subsection{Dataset Overview}
    
        The dataset consists of logs from SSH attacks. Each entry contains timestamps, intents, and textual descriptions of attack sessions. Key features include:
        
        \begin{itemize}
            \item Timestamps: Indicating when the attack occurred.
            \item Intents: Representing the purpose or action of the attacker.
            \item Session Data: Textual information about each attack session.
        \end{itemize}

    \subsection{Dataset Preparation}
    
        The dataset used in this research is loaded from a Parquet file (\texttt{ssh\_attacks.parquet}) into a Pandas DataFrame. The initial inspection involves checking the dataset's structure, identifying missing values, and detecting duplicate rows. The dataset contains columns such as \texttt{Session ID}, \texttt{Full session text}, \texttt{Timestamp}, and \texttt{Intent labels}.
        
        \textbf{Refer to Appendix \ref{lst:load-inspect-dataset} for the code snippet.}

        The initial inspection revealed that the dataset is well-structured with columns that are essential for our analysis. However, it is important to handle any missing values and duplicates to ensure the integrity of the data. The following steps were taken to address these issues:

        \begin{itemize}
            \item **Missing Values**: We identified and handled missing values by either imputing them with appropriate values or removing the affected rows.
            \item **Duplicate Rows**: Duplicate rows were detected and removed to avoid redundancy in the analysis.
        \end{itemize}

        \textbf{Table \ref{tab:dataset-structure}} summarizes the dataset structure.

        \begin{table}[h]
            \centering
            \begin{tabular}{|l|l|}
                \hline
                \textbf{Column} & \textbf{Description} \\ \hline
                Session ID & Unique identifier for each session \\ \hline
                Full session text & Text of the entire session \\ \hline
                Timestamp & Timestamp of the session \\ \hline
                Intent labels & Labels indicating the intent of the session \\ \hline
            \end{tabular}
            \vspace{1em}
            \caption{Dataset Structure}
            \label{tab:dataset-structure}
        \end{table}

    \subsection{Temporal Analysis}
    
        The temporal analysis examines when the attacks were performed. The \texttt{first\_timestamp} column is converted to a datetime format to analyze attack frequencies over time, including hourly, daily, and monthly trends.
            
        \textbf{Refer to Appendix \ref{lst:convert-analyze-frequencies} for the code snippet.}

        The analysis includes plotting the number of attacks per hour, month, and year to identify patterns and trends. This helps in understanding the temporal distribution of attacks and identifying any periodic patterns or anomalies.

        \textbf{Figure \ref{fig:temporal-analysis}} shows the temporal analysis of SSH attacks.

        \begin{figure}[h]
            \centering
            \includegraphics[width=0.8\textwidth]{../figures/plots/section1/temporal_series_of_ssh_attacks.png}
            \caption{Temporal Series of SSH Attacks}
            \label{fig:temporal-analysis}
        \end{figure}

        The temporal analysis revealed that the frequency of attacks varies significantly over time. By examining the hourly, daily, and monthly trends, we can gain insights into the behavior of attackers and the times when systems are most vulnerable.

    \subsection{Feature Extraction}
    
        Feature extraction involves identifying and extracting relevant features from the attack sessions. This includes analyzing the distribution of classes (intents) and visualizing the data using bar plots.
            
        \textbf{Refer to Appendix \ref{lst:extract-visualize-classes} for the code snippet.}

        The distribution of classes provides valuable information about the types of attacks and their prevalence. By visualizing this distribution, we can identify the most common attack intents and focus our analysis on these areas.

        \textbf{Figure \ref{fig:class-distribution}} shows the distribution of classes.

        \begin{figure}[h]
            \centering
            \includegraphics[width=0.8\textwidth]{../figures/plots/section1/distribution_of_classes.png}
            \caption{Distribution of Classes}
            \label{fig:class-distribution}
        \end{figure}

        The feature extraction process also involves creating new features that can enhance the analysis. For example, we can extract the length of each session, the number of commands executed, and other relevant metrics.

    \subsection{Common Words Analysis}
    
        The common words analysis identifies the most frequent words used in the attack sessions. This is achieved using word clouds and other text analysis techniques.
            
        \textbf{Refer to Appendix \ref{lst:generate-wordcloud} for the code snippet.}

        The word cloud visualization highlights the most common words used in the attack sessions, providing insights into the attackers' behavior and strategies. This can help in identifying common commands and patterns used in the attacks.

        \textbf{Figure \ref{fig:word-cloud}} shows the word cloud of the most common words.

        \begin{figure}[h]
            \centering
            \includegraphics[width=0.8\textwidth]{../figures/plots/section1/word_cloud_of_most_common_words.png}
            \caption{Word Cloud of Most Common Words}
            \label{fig:word-cloud}
        \end{figure}

        Additionally, we can create bar plots to show the frequency of the top 10 most common words, which can further aid in understanding the textual characteristics of the attack sessions.

        \textbf{Figure \ref{fig:common-words}} shows the top 10 most common words.

        \begin{figure}[h]
            \centering
            \includegraphics[width=0.8\textwidth]{../figures/plots/section1/top_10_most_common_words.png}
            \caption{Top 10 Most Common Words}
            \label{fig:common-words}
        \end{figure}

    \subsection{Intent Distribution}
            
        The intent distribution analysis examines the distribution of intents over time. This involves grouping the data by date and intent to count occurrences and visualize the trends.
            
        \textbf{Refer to Appendix \ref{lst:group-attacks} for the code snippet.}

        By analyzing the distribution of intents over time, we can identify trends and patterns in the attackers' behavior. This can help in understanding how different types of attacks evolve and vary over time.

        \textbf{Figure \ref{fig:intent-distribution}} shows the distribution of intents over time.

        \begin{figure}[h]
            \centering
            \includegraphics[width=0.8\textwidth]{../figures/plots/section1/intents_over_timestamps.png}
            \caption{Distribution of Intents Over Time}
            \label{fig:intent-distribution}
        \end{figure}

        The intent distribution analysis also helps in identifying any seasonal or periodic patterns in the attacks, which can be crucial for developing effective defense strategies.

    \subsection{Text Representation}
    
        Text representation converts the session text into numerical representations using techniques such as Bag of Words (BoW) and Term Frequency-Inverse Document Frequency (TF-IDF). These representations are used for further analysis and machine learning tasks.
        
        \textbf{Refer to Appendix \ref{lst:convert-text-numerical} for the code snippet.}

        The resulting numerical representations from both BoW and TF-IDF are normalized and used for subsequent analysis and modeling. These representations are essential for applying machine learning algorithms to classify and predict attack intents.

        \textbf{Table \ref{tab:text-representation}} summarizes the text representation techniques.

        \begin{table}[h]
            \centering
            \caption{Text Representation Techniques}
            \label{tab:text-representation}
            \begin{tabular}{|l|l|}
                \hline
                \textbf{Technique} & \textbf{Description} \\ \hline
                Bag of Words (BoW) & Converts text into a matrix of token counts \\ \hline
                TF-IDF & Converts text into a matrix of TF-IDF features \\ \hline
            \end{tabular}
        \end{table}

        The text representation techniques help in transforming the unstructured session text into structured numerical data, which can be used for various analytical and predictive tasks. By comparing the performance of different representation techniques, we can select the most effective method for our analysis.

% 04-supervised-learning-classification.tex

% Supervised Learning – Classification
% 4.1. Introduction: Provides an overview of the supervised learning task and its objectives.
% 4.2. Data Splitting: Describes the process of splitting the dataset into training and test sets.
% 4.3. Baseline Model Implementation: Implements and evaluates baseline models.
% 4.4. Hyperparameter Tuning: Tunes hyperparameters and evaluates performance.
% 4.5. Result Analysis: Analyzes the results for each intent.
% 4.6. Feature Experimentation: Explores different feature combinations and their impact on performance.

% Section Title
\section{SUPERVISED LEARNING - CLUSTERING}

    % Main Content

    \subsection{Introduction}
    
        This section provides an overview of the supervised learning task and its objectives. The goal is to classify attack session tactics based on the provided dataset. We will implement and evaluate various machine learning models to determine the most effective approach for this classification task.

    \subsection{Data Splitting}
    
        The first step in the supervised learning process is to split the dataset into training and test sets. This ensures that we can evaluate the performance of our models on unseen data.

        \textbf{Data Loading:} The dataset is loaded from a Parquet file into a Pandas DataFrame.

        \begin{verbatim}
            # Load the dataset
            SSH_Attacks = pd.read_parquet("../data/processed/ssh_attacks_decoded.parquet")
        \end{verbatim}

        \textbf{Data Splitting:} We split the dataset into training and test sets, ensuring a 70/30 split while maintaining reproducibility.

        \begin{verbatim}
            # Split the dataset into training and test sets
            X_train, X_test, y_train, y_test = train_test_split(
                X, y, test_size=0.3, random_state=42
            )
        \end{verbatim}

        \textbf{Placeholder for Data Splitting Summary Table}

        The data splitting process ensures that the training set is used to train the models, while the test set is used to evaluate their performance.
            
    \subsection{Baseline Model Implementation}
    
        In this subsection, we implement and evaluate baseline models to establish a performance benchmark. We will use Logistic Regression, Random Forest, and Support Vector Machine (SVM) as our baseline models.

        \textbf{Logistic Regression:}

        \begin{verbatim}
            # Initialize and train Logistic Regression model
            model = LogisticRegression(max_iter=1000, random_state=42)
            model.fit(X_train_tfidf, y_train_binary)
        \end{verbatim}

        \textbf{Random Forest:}

        \begin{verbatim}
            # Initialize and train Random Forest model
            model = RandomForestClassifier(n_estimators=100, random_state=42)
            model.fit(X_train_tfidf, y_train_binary)
        \end{verbatim}

        \textbf{Support Vector Machine (SVM):}

        \begin{verbatim}
            # Initialize and train SVM model
            model = SVC(kernel='linear', random_state=42)
            model.fit(X_train_tfidf, y_train_binary)
        \end{verbatim}

        \textbf{Placeholder for Baseline Model Performance Table}

        The baseline model implementation provides a reference point for evaluating the performance of more advanced models.
            
    \subsection{Hyperparameter Tuning}
    
        Hyperparameter tuning involves optimizing the parameters of the models to improve their performance. We use GridSearchCV to perform an exhaustive search over specified parameter values.

        \textbf{Logistic Regression Hyperparameter Tuning:}

        \begin{verbatim}
            # Define parameter grid for Logistic Regression
            param_grid = {'C': [0.1, 1, 10, 100]}
            grid_search = GridSearchCV(LogisticRegression(max_iter=1000, random_state=42), param_grid, cv=5)
            grid_search.fit(X_train_tfidf, y_train_binary)
        \end{verbatim}

        \textbf{Random Forest Hyperparameter Tuning:}

        \begin{verbatim}
            # Define parameter grid for Random Forest
            param_grid = {'n_estimators': [50, 100, 200]}
            grid_search = GridSearchCV(RandomForestClassifier(random_state=42), param_grid, cv=5)
            grid_search.fit(X_train_tfidf, y_train_binary)
        \end{verbatim}

        \textbf{Placeholder for Hyperparameter Tuning Results Table}

        Hyperparameter tuning helps in finding the best parameters for each model, thereby improving their performance.
            
    \subsection{Result Analysis}
    
        In this subsection, we analyze the results of the models for each intent. We use metrics such as accuracy, precision, recall, and F1-score to evaluate the performance.

        \textbf{Classification Report:}

        \begin{verbatim}
            # Generate classification report
            report = classification_report(y_test_binary, y_pred, zero_division=0)
            print(report)
        \end{verbatim}

        \textbf{Confusion Matrix:}

        \begin{verbatim}
            # Generate confusion matrix
            cm = confusion_matrix(y_test_binary, y_pred)
            sns.heatmap(cm, annot=True, fmt='d', cmap='coolwarm')
            plt.show()
        \end{verbatim}

        \textbf{Placeholder for Classification Report and Confusion Matrix Plots}

        The result analysis provides insights into the performance of the models and helps in identifying areas for improvement.
            
    \subsection{Feature Experimentation}
    
        Feature experimentation involves exploring different feature combinations and their impact on model performance. We experiment with various text representation techniques such as Bag of Words (BoW) and Term Frequency-Inverse Document Frequency (TF-IDF).

        \textbf{Bag of Words (BoW):}

        \begin{verbatim}
            # Convert text into numerical representations using Bag of Words (BoW)
            bow_vectorizer = CountVectorizer()
            X_train_bow = bow_vectorizer.fit_transform(X_train)
            X_test_bow = bow_vectorizer.transform(X_test)
        \end{verbatim}

        \textbf{TF-IDF:}

        \begin{verbatim}
            # Convert text into numerical representations using TF-IDF
            tfidf_vectorizer = TfidfVectorizer()
            X_train_tfidf = tfidf_vectorizer.fit_transform(X_train)
            X_test_tfidf = tfidf_vectorizer.transform(X_test)
        \end{verbatim}

        \textbf{Placeholder for Feature Experimentation Results Table}

        By experimenting with different features, we can identify the most effective representation techniques for our classification task.

% 05-unsupervised-learning-clustering.tex

% Unsupervised Learning – Clustering
% 5.1. Introduction: Provides an overview of the clustering task and its objectives.
% 5.2. Determine the Number of Clusters: Uses methods like the elbow method or silhouette analysis to determine the number of clusters.
% 5.3. Hyperparameter Tuning: Tunes other hyperparameters, if any.
% 5.4. Cluster Visualization: Visualizes the clusters through t-SNE.
% 5.5. Cluster Analysis: Analyzes the characteristics of each cluster.
% 5.6. Intent Homogeneity: Assesses if clusters reflect intent division.
% 5.7. Specific Attack Categories: Associates clusters with specific attack categories.

% Section Title
\section{UNSUPERVISED LEARNING - CLUSTERING}

    % Main Content

    \subsection{Introduction}
    
        This section of the project focuses on applying unsupervised learning techniques to analyze SSH attack data. Specifically, clustering methods were utilized to uncover patterns and relationships within the dataset. An emphasis was placed on identifying distinct attack behaviors and intents.

    \subsection{Data Preparation}
    
        The dataset chosen was the one generated through the TF-IDF vectorization technique. This was made because it was essential to start with a dataset that represented in the best way the frequency and the importance of words, making each word as a dimension of our vector.

    \subsection{Clustering Methods}
    
        Two clustering techniques were applied:
        
        \begin{itemize}
        
            \item \textbf{K-Means Clustering}: Using the Elbow Method and silhouette scores, the optimal number of clusters was determined.
            
            \item \textbf{Gaussian Mixture Model (GMM)}: Similar to K-Means, the optimal number of clusters was selected using silhouette scores and log-likelihood values.
            
        \end{itemize}

    \subsection{Hyperparameter Tuning}
    
        Hyperparameter tuning was conducted to optimize the performance of both clustering methods. For K-Means, parameters such as the initialization method (\texttt{k-means++} and \texttt{random}), the number of initializations, and the maximum number of iterations were fine-tuned using a grid search approach. Similarly, GMM parameters including the initialization method (\texttt{kmeans}), covariance type (\texttt{full} and \texttt{spherical}), and tolerance were optimized. These steps ensured the models were tailored to the dataset, resulting in better clustering outcomes.

    \subsection{Clusters Visualization}
    
        To visualize the clustering results, t-SNE dimensionality reduction was applied. Its functionalities makes it an excellent choice for visualizing clusters in datasets where direct interpretation is difficult due to high dimensionality. By projecting the data into a two-dimensional space, t-SNE enables us to identify patterns and groupings that may not be evident in the original feature space. The two-dimensional plots provided a clear representation of the clusters formed by both K-Means and GMM, highlighting their separability and internal consistency.
        
        \begin{figure}[H]
            \centering
            \begin{subfigure}[c]{0.47\textwidth}
                \centering
                \includegraphics[width=\textwidth]{../figures/plots/section3/tsne_kmeans_clusters.png}
                \caption{t-SNE Visualization of K-Means Clusters.}
                \label{fig:tsne_kmeans}
            \end{subfigure}
            \hfill
            \begin{subfigure}[c]{0.47\textwidth}
                \centering
                \includegraphics[width=\textwidth]{../figures/plots/section3/tsne_gmm_clusters.png}
                \caption{t-SNE Visualization of GMM Clusters.}
                \label{fig:tsne_gmm}
            \end{subfigure}
            \vspace{-0.1cm}
            \caption{}
            \label{fig:}
        \end{figure}

        \subsubsection{K-Means Visualization \\}
        
            % text

        \subsubsection{GMM Visualization \\}

            % text

    \subsection{Clusters Analysis}

        \subsubsection{Word Cloud Representation \\}

            Word clouds were generated for each cluster to highlight the most significant terms. This provided an intuitive understanding of the key features within each cluster, offering insights into the behavioral patterns of the attacks.

            \begin{figure}[H]
                \centering
                \includegraphics[width=0.9\textwidth]{../figures/plots/section3/circular_wordclouds.png}
                \caption{Word Clouds for Each Cluster.}
                \label{fig:word_clouds}
            \end{figure}

        \subsubsection{Community Detection \\}

            Graph-based community detection was performed within selected clusters to identify subgroups. The analysis revealed meaningful relationships and substructures within the data, as demonstrated in the visualizations below:
            
            \begin{figure}[H]
                \centering
                \begin{subfigure}[c]{0.47\textwidth}
                    \centering
                    \includegraphics[width=\textwidth]{../figures/plots/section3/k-means_graph_visualization_of_cluster_0_with_communities.png}
                    \caption{Community Detection in Cluster 0 (K-Means).}
                    \label{fig:kmeans_graph}
                \end{subfigure}
                \hfill
                \begin{subfigure}[c]{0.47\textwidth}
                    \centering
                    \includegraphics[width=\textwidth]{../figures/plots/section3/gmm_graph_visualization_of_cluster_0_with_communities.png}
                    \caption{Community Detection in Cluster 0 (GMM).}
                    \label{fig:gmm_graph}
                \end{subfigure}
                \vspace{-0.1cm}
                \caption{}
                \label{fig:}
            \end{figure}

            % text

    \subsection{Conclusion}
    
        This analysis provided valuable insights into the patterns of SSH attacks through clustering. The K-Means and GMM algorithms both effectively identified meaningful clusters, as supported by validation metrics and visualizations. Hyperparameter tuning further enhanced the performance of both models. The results demonstrate the potential of unsupervised learning in uncovering hidden patterns in complex datasets, providing a foundation for future applications such as anomaly detection and improved cybersecurity strategies.

% 06-language-model-exploration.tex

% Language Models Exploration
% 6.1. Introduction: Provides an overview of the language models task and its objectives.
% 6.2. Pretraining: Describes the process of pretraining Doc2Vec or using a pretrained Bert model.
% 6.3. Model Fine-tuning: Fine-tunes the last layer of the network.
% 6.4. Learning Curves: Plots learning curves and determines the optimal number of epochs.

% Section Title
\section{LANGUAGE MODEL EXPLORATION}

    % Main Content

    \subsection{Introduction}
    
        This section provides an overview of the language models task and its objectives. The goal is to leverage advanced language models to classify attack session tactics based on the provided dataset. We will explore the use of pretrained models such as BERT and Doc2Vec, fine-tune them for our specific task, and analyze their performance.

        Language models have revolutionized natural language processing (NLP) by enabling transfer learning, where models pretrained on large datasets can be fine-tuned on specific tasks. This approach allows us to benefit from the knowledge encoded in these models and achieve better performance with less training data.

    \subsection{Pretraining}
    
        Pretraining involves using a pretrained language model or training a model from scratch on a large corpus of text. For this task, we will use a pretrained BERT model from HuggingFace's Transformers library.

        \textbf{Installing Dependencies:}

        % Install required packages
        \begin{lstlisting}[language=bash, caption={Install required packages}, label={lst:install_packages}]
            !pip install transformers torch
        \end{lstlisting}
        
        \vspace{1em}

        \textbf{Loading the Dataset:}

        % Load dataset and print its size
        \begin{lstlisting}[caption={Load dataset and print its size}, label={lst:load_dataset}]
            import pandas as pd

            # Load the dataset
            df = pd.read_parquet("../data/processed/ssh_attacks_sampled_decoded.parquet")
            print(f"Dataset size: {df.shape[0]} rows")
        \end{lstlisting}
        
        \vspace{1em}

        \textbf{Preprocessing:}

        % Preprocess `Set_Fingerprint` column
        \begin{lstlisting}[caption={Preprocess `Set\_Fingerprint` column}, label={lst:preprocess-fingerprint}]
            from sklearn.preprocessing import MultiLabelBinarizer

            # Preprocess Set_Fingerprint column
            df['Set_Fingerprint'] = df['Set_Fingerprint'].apply(lambda x: [intent.strip() for intent in x.split(',')])
            mlb = MultiLabelBinarizer()
            y = mlb.fit_transform(df['Set_Fingerprint'])
            print(f"Classes identified: {mlb.classes_}")
        \end{lstlisting}
        
        \vspace{1em}

        \textbf{Placeholder for Dataset Summary Table}
        
    \subsection{Model Fine-tuning}
    
        Fine-tuning involves training the last layer of the pretrained model on our specific dataset. We will use BERT for sequence classification and fine-tune it on the SSH attack sessions.

        \textbf{Tokenization:}

        % Tokenize text data using BERT tokenizer
        \begin{lstlisting}[caption={Tokenize text data using BERT tokenizer}, label={lst:bert_tokenizer}]
            from transformers import BertTokenizer

            # Tokenize the text data
            tokenizer = BertTokenizer.from_pretrained('bert-base-uncased')
            train_encodings = tokenizer(list(train_texts.fillna("").astype(str)), truncation=True, padding=True, max_length=128)
            val_encodings = tokenizer(list(val_texts.fillna("").astype(str)), truncation=True, padding=True, max_length=128)
        \end{lstlisting}
        
        \vspace{1em}

        \textbf{Model Initialization:}

        % Initialize BERT model for sequence classification
        \begin{lstlisting}[caption={Initialize BERT model for sequence classification}, label={lst:bert_model}]
            from transformers import BertForSequenceClassification, AdamW

            # Initialize the BERT model for sequence classification
            model = BertForSequenceClassification.from_pretrained('bert-base-uncased', num_labels=y.shape[1])
            model.to(device)

            # Optimizer and Loss
            optimizer = AdamW(model.parameters(), lr=5e-5)
            criterion = torch.nn.BCEWithLogitsLoss()
        \end{lstlisting}
        
        \vspace{1em}

        \textbf{Training Loop:}

        % Fine-tune BERT model
        \begin{lstlisting}[caption={Fine-tune BERT model}, label={lst:bert_fine_tune}]
            train_loss_list, val_loss_list = [], []

            for epoch in range(5):  # Fine-tune for 5 epochs
                model.train()
                total_loss = 0

                for batch in train_loader:
                    optimizer.zero_grad()
                    input_ids, attention_mask, labels = (
                        batch['input_ids'].to(device),
                        batch['attention_mask'].to(device),
                        batch['labels'].to(device),
                    )
                    outputs = model(input_ids=input_ids, attention_mask=attention_mask)
                    loss = criterion(outputs.logits, labels)
                    loss.backward()
                    optimizer.step()
                    total_loss += loss.item()

                train_loss_list.append(total_loss / len(train_loader))

                # Validation
                model.eval()
                val_loss = 0
                with torch.no_grad():
                    for batch in val_loader:
                        input_ids, attention_mask, labels = (
                            batch['input_ids'].to(device),
                            batch['attention_mask'].to(device),
                            batch['labels'].to(device),
                        )
                        outputs = model(input_ids=input_ids, attention_mask=attention_mask)
                        loss = criterion(outputs.logits, labels)
                        val_loss += loss.item()
                val_loss_list.append(val_loss / len(val_loader))
        \end{lstlisting}
        
        \vspace{1em}

        \textbf{Placeholder for Training and Validation Loss Table}
            
    \subsection{Learning Curves}
    
        Plotting learning curves helps in understanding the model's performance over epochs and determining the optimal number of epochs for training.

        \textbf{Plotting Learning Curves:}

        % Plot learning curves
        \begin{lstlisting}[caption={Plot learning curves}, label={lst:plot_learning_curves}]
            import matplotlib.pyplot as plt

            # Plot learning curves
            plt.plot(range(1, 6), train_loss_list, label="Training Loss")
            plt.plot(range(1, 6), val_loss_list, label="Validation Loss")
            plt.xlabel("Epochs")
            plt.ylabel("Loss")
            plt.legend()
            plt.show()
        \end{lstlisting}
        
        \vspace{1em}

        \textbf{Placeholder for Learning Curves Plot}

        By analyzing the learning curves, we can determine the optimal number of epochs to stop training and avoid overfitting. The point where the validation loss stops decreasing or starts increasing indicates the optimal stopping point.

% 07-conclusion.tex

% Conclusion: Summarizing the key findings, challenges, and future work.

% Section Title
\section{CONCLUSION}

    % Main Content

    \subsection{Summary of Key Findings}
    
        In this project, we explored various techniques for analyzing and classifying SSH shell attack logs. The primary objectives were to preprocess the data, perform exploratory data analysis, implement supervised and unsupervised learning models, and leverage advanced language models for classification tasks. Here, we summarize the key findings from each section of the project.

        \textbf{Data Exploration and Pre-processing:} We began by loading and inspecting the dataset, identifying missing values, and handling duplicates. Temporal analysis revealed significant variations in attack frequencies over time, with notable peaks during specific hours and months. Feature extraction and common words analysis provided insights into the most frequent commands and intents used in the attack sessions.

        \textbf{Supervised Learning - Classification:} We implemented and evaluated several machine learning models, including Logistic Regression, Random Forest, and Support Vector Machine (SVM). Hyperparameter tuning improved the performance of these models, and the result analysis highlighted the strengths and weaknesses of each approach. Feature experimentation with different text representation techniques, such as Bag of Words (BoW) and TF-IDF, demonstrated the impact of feature selection on model performance.

        \textbf{Unsupervised Learning - Clustering:} Clustering techniques, such as K-Means and Gaussian Mixture Models (GMM), were used to group similar attack sessions. The elbow method and silhouette analysis helped determine the optimal number of clusters. Cluster visualization using t-SNE provided a clear representation of the clusters, and cluster analysis revealed common patterns and behaviors within each group.

        \textbf{Language Model Exploration:} We explored the use of advanced language models, such as BERT, for classifying attack session tactics. Fine-tuning the pretrained BERT model on our dataset improved classification performance. Learning curves indicated the optimal number of epochs for training, helping to avoid overfitting.

    \subsection{Challenges Faced}
    
        Throughout the project, we encountered several challenges that required careful consideration and problem-solving.

        \textbf{Data Quality and Preprocessing:} Handling missing values, duplicates, and inconsistencies in the dataset was a critical step. Ensuring the data was clean and well-prepared for analysis required significant effort. Additionally, the unstructured nature of the session text posed challenges for text representation and feature extraction.

        \textbf{Model Selection and Tuning:} Selecting appropriate machine learning models and tuning their hyperparameters was a complex task. Balancing model complexity with performance and avoiding overfitting required iterative experimentation and validation.

        \textbf{Computational Resources:} Training advanced language models, such as BERT, required substantial computational resources. Efficiently managing these resources and optimizing the training process was essential to achieve timely results.

        \textbf{Interpretability of Results:} Interpreting the results of clustering and classification models, especially in the context of cybersecurity, was challenging. Ensuring that the findings were meaningful and actionable required careful analysis and domain knowledge.

    \subsection{Future Work}
    
        Based on the findings and challenges encountered in this project, we propose several directions for future work.

        \textbf{Enhanced Feature Engineering:} Further exploration of feature engineering techniques, such as incorporating domain-specific knowledge and using advanced text representation methods, could improve model performance. Experimenting with additional features, such as network metadata and contextual information, may provide deeper insights into attack patterns.

        \textbf{Advanced Model Architectures:} Exploring more advanced model architectures, such as transformer-based models and deep neural networks, could enhance classification accuracy. Transfer learning with other pretrained models and ensemble methods may also yield better results.

        \textbf{Real-time Analysis and Detection:} Implementing real-time analysis and detection systems for SSH shell attacks could provide immediate insights and responses to potential threats. Integrating the models developed in this project into a real-time monitoring framework would be a valuable extension.

        \textbf{Broader Dataset and Generalization:} Expanding the dataset to include a wider range of attack types and sources would improve the generalizability of the models. Collaborating with other organizations to share data and insights could enhance the robustness and applicability of the findings.

    \subsection{Conclusion}
    
        This project demonstrated the potential of machine learning and advanced language models for analyzing and classifying SSH shell attack logs. By leveraging various techniques, we gained valuable insights into attack patterns and behaviors, which can inform cybersecurity strategies and defenses. Despite the challenges faced, the results highlight the importance of data-driven approaches in enhancing cybersecurity threat detection and response capabilities. Future work in this area holds promise for further advancements and practical applications in the field of cybersecurity.


% Include the appendix
% % appendix.tex

\appendix

\section{Appendix}

    \subsection{Code Snippets}

        \subsubsection{Data Exploration and Pre-processing}

            % Load and inspect the dataset
            \begin{lstlisting}[caption={Load and inspect the dataset}, label={lst:load-inspect-dataset}]
                # Load the dataset
                SSH_Attacks = pd.read_parquet("../data/processed/ssh_attacks_decoded.parquet")
        
                # Inspect the dataset structure
                print(SSH_Attacks.info())
        
                # Check for missing values
                print(SSH_Attacks.isnull().sum())
        
                # Check for duplicate rows
                print(SSH_Attacks.duplicated().sum())
            \end{lstlisting}

            % Convert timestamps and analyze frequencies
            \begin{lstlisting}[caption={Convert timestamps and analyze frequencies}, label={lst:convert-analyze-frequencies}]
                # Convert first_timestamp to datetime format
                SSH_Attacks['first_timestamp'] = pd.to_datetime(SSH_Attacks['first_timestamp'])

                # Analyze attack frequencies over time
                temporal_series = (
                    SSH_Attacks.groupby(SSH_Attacks['first_timestamp'].dt.date)
                    .size()
                    .reset_index(name='attack_count')
                )
            \end{lstlisting}

            % Extract and visualize class distribution
            \begin{lstlisting}[caption={Extract and visualize class distribution}, label={lst:extract-visualize-classes}]
                # Extract and count occurrences of each class
                all_classes = SSH_Attacks['Set_Fingerprint'].explode().str.strip()
                class_counts = all_classes.value_counts()

                # Plot the distribution of classes
                sns.barplot(x=class_counts.index, y=class_counts.values, palette='viridis')
            \end{lstlisting}

            % Generate a word cloud from session text
            \begin{lstlisting}[caption={Generate a word cloud from session text}, label={lst:generate-wordcloud}]
                # Generate a word cloud for the session text
                wordcloud = WordCloud(width=800, height=400, background_color='white').generate(' '.join(SSH_Attacks['Full session text']))
                plt.imshow(wordcloud, interpolation='bilinear')
                plt.axis('off')
                plt.show()
            \end{lstlisting}

            % Group attacks by fingerprint and date
            \begin{lstlisting}[caption={Group attacks by fingerprint and date}, label={lst:group-attacks}]
                # Group by Set_Fingerprint and date to count occurrences
                grouped_SSH_Attacks = (
                    SSH_Attacks.explode('Set_Fingerprint')
                    .groupby([SSH_Attacks['first_timestamp'].dt.date, 'Set_Fingerprint'])
                    .size()
                    .reset_index(name='attack_count')
                )
            \end{lstlisting}

            % Convert text into numerical representations
            \begin{lstlisting}[caption={Convert text into numerical representations}, label={lst:convert-text-numerical}]
                # Convert text into numerical representations using Bag of Words (BoW)
                from sklearn.feature_extraction.text import CountVectorizer
                bow_vectorizer = CountVectorizer()
                X_bow = bow_vectorizer.fit_transform(SSH_Attacks['Full session text'])

                # Convert text into numerical representations using TF-IDF
                from sklearn.feature_extraction.text import TfidfVectorizer
                tfidf_vectorizer = TfidfVectorizer()
                X_tfidf = tfidf_vectorizer.fit_transform(SSH_Attacks['Full session text'])
            \end{lstlisting}

        \subsubsection{Supervised Learning - Classification}

            % TODO: add

        \subsubsection{Unsupervised Learning - Clustering}

            % Elbow Method for k-Means Clustering
            \begin{lstlisting}[caption={Elbow Method for k-Means Clustering}, label={lst:elbow_method}]
                # Elbow Method
                n_cluster_list = []
                inertia_list = []
                for n_clusters in range(3, 17):
                    kmeans = KMeans(n_clusters=n_clusters, n_init=10, random_state=42)
                    kmeans.fit(X)
                    inertia_list.append(kmeans.inertia_)
                    n_cluster_list.append(n_clusters)
                
                # Plot Elbow Method
                plt.figure(figsize=(5, 3.5))
                plt.plot(n_cluster_list, inertia_list, marker='o', markersize=5, color='blue')
                plt.xlabel('Number of clusters')
                plt.ylabel('k-Means clustering error')
                plt.title('Elbow Method')
                plt.show()
            \end{lstlisting}

            % Silhouette Analysis for k-Means Clustering
            \begin{lstlisting}[caption={Silhouette Analysis for k-Means Clustering}, label={lst:silhouette_analysis}]
                # Silhouette Analysis
                silhouette_list = []
                for n_clusters in range(3, 17):
                    kmeans = KMeans(n_clusters=n_clusters, n_init=10, random_state=42)
                    labels = kmeans.fit_predict(X)
                    silhouette_score_value = silhouette_score(X, labels)
                    silhouette_list.append(silhouette_score_value)
                
                # Plot Silhouette Analysis
                plt.figure(figsize=(5, 3.5))
                plt.plot(n_cluster_list, silhouette_list, marker='o', markersize=5, color='blue')
                plt.xlabel('Number of clusters')
                plt.ylabel('Silhouette Score')
                plt.title('Silhouette Analysis')
                plt.show()
            \end{lstlisting}

            % Grid Search for k-Means Clustering
            \begin{lstlisting}[caption={Grid Search for k-Means Clustering}, label={lst:grid_search_kmeans}]
                # Define parameter grid for K-Means
                param_grid_kmeans = {
                    'init': ['k-means++', 'random'],
                    'n_init': list(range(10, 21, 2)),
                    'max_iter': list(range(50, 200, 50)),
                }
                
                # Create KMeans object
                kmeans = KMeans(n_clusters=10, random_state=42)
                
                # Create GridSearchCV object
                grid_search_kmeans = GridSearchCV(kmeans, param_grid=param_grid_kmeans, cv=5)
                
                # Fit the grid search to the data
                grid_search_kmeans.fit(X)
                
                # Get the best parameters
                best_params_kmeans = grid_search_kmeans.best_params_
                print("Best parameters:", best_params_kmeans)
            \end{lstlisting}

            % Grid Search for Gaussian Mixture Model (GMM)
            \begin{lstlisting}[caption={Grid Search for Gaussian Mixture Model (GMM)}, label={lst:grid_search_gmm}]
                # Define parameter grid for GMM
                param_grid_gmm = {
                    'init_params': ['kmeans'],
                    'covariance_type': ['full', 'spherical'],
                    'tol': [1e-3, 1e-4, 1e-5],
                    'max_iter': list(range(50, 300, 50)),
                }
                
                # Create GaussianMixture object
                gmm = GaussianMixture(n_components=10, random_state=42)
                
                # Create GridSearchCV object
                grid_search_gmm = GridSearchCV(gmm, param_grid=param_grid_gmm, cv=5, scoring=silhouette_scorer)
                
                # Fit the grid search to the data
                grid_search_gmm.fit(X)
                
                # Get the best parameters
                best_params_gmm = grid_search_gmm.best_params_
                print("Best parameters:", best_params_gmm)
            \end{lstlisting}

            % t-SNE Visualization of Clusters
            \begin{lstlisting}[caption={t-SNE Visualization of Clusters}, label={lst:tsne_visualization}]
                # Apply t-SNE to reduce the number of components
                tsne = TSNE(n_components=2, random_state=42).fit_transform(X)
                df_tsne = pd.DataFrame(tsne, columns=['x1', 'x2'])
                
                # K-Means Clusters
                df_tsne['cluster_kmeans'] = kmeans_tuned.labels_
                sns.scatterplot(data=df_tsne, x='x1', y='x2', hue='cluster_kmeans', palette='viridis')
                plt.title('t-SNE Visualization of K-Means Clusters')
                plt.show()
                
                # GMM Clusters
                df_tsne['cluster_gmm'] = gmm_tuned.predict(X)
                sns.scatterplot(data=df_tsne, x='x1', y='x2', hue='cluster_gmm', palette='viridis')
                plt.title('t-SNE Visualization of GMM Clusters')
                plt.show()
            \end{lstlisting}

            % Feature Distribution Analysis by Cluster
            \begin{lstlisting}[caption={Feature Distribution Analysis by Cluster}, label={lst:feature_distribution}]
                # Analyze the distribution of features within each cluster
                for cluster in range(10):
                    cluster_data = df_tsne[df_tsne['cluster_kmeans'] == cluster]
                    print(f"Cluster {cluster} Feature Distribution:")
                    print(cluster_data.describe())
            \end{lstlisting}

            % Intent Proportions Analysis by Cluster
            \begin{lstlisting}[caption={Intent Proportions Analysis by Cluster}, label={lst:intent_proportions}]
                # Calculate the proportion of each intent within the clusters
                for cluster in range(10):
                    cluster_data = df_tsne[df_tsne['cluster_kmeans'] == cluster]
                    intent_proportions = cluster_data['intent'].value_counts(normalize=True)
                    print(f"Cluster {cluster} Intent Proportions:")
                    print(intent_proportions)
            \end{lstlisting}

            % Attack Categories Analysis by Cluster
            \begin{lstlisting}[caption={Attack Categories Analysis by Cluster}, label={lst:attack_categories}]
                # Analyze the most frequent attack categories within the clusters
                for cluster in range(10):
                    cluster_data = df_tsne[df_tsne['cluster_kmeans'] == cluster]
                    attack_categories = cluster_data['attack_category'].value_counts()
                    print(f"Cluster {cluster} Attack Categories:")
                    print(attack_categories)
            \end{lstlisting}

        \subsubsection{Language Model Exploration}

            % TODO: add


% TODO: check
% Include the bibliography file
\bibliography{../template/column-format-template/sample-base}

% At end of document
% \bibliography{references}           % Single .bib file
% \bibliography{ref1,ref2,ref3}       % Multiple .bib files

\end{document}
