% 01-introduction.tex

% Introduction: Provides an overview of the project, its objectives, and the significance of analyzing SSH shell attack logs.

\vspace{-1cm}

% Section Title
\section{INTRODUCTION}

    % Main Content
    % This section introduces the topic of the project, provides background information, and outlines the objectives.

    \subsection{Motivation}

        Security logs play a crucial role in understanding and mitigating cyber attacks, particularly in the domain of network and system security. With the increasing sophistication of cyber threats, analyzing and interpreting security logs has become paramount for detecting, preventing, and responding to potential security breaches. Unix shell attacks, especially those executed through SSH, represent a significant vector for potential system compromises.

        The complexity of security log analysis stems from several key challenges:
        
        \begin{itemize}
            \item Logs are often unstructured and contain ambiguous or malformed text.
            \item Manual parsing and interpretation of logs is time-consuming and error-prone.
            \item The sheer volume of log data makes comprehensive manual review impractical.
        \end{itemize}

        These challenges underscore the need for automated, intelligent approaches to log analysis that can efficiently extract meaningful insights and identify potential security threats.

    \subsection{Objective}

        The primary objective of this research is to develop and evaluate machine learning techniques for automatic analysis and classification of SSH shell attack logs. Specifically, we aim to:
        
        \begin{itemize}
            \item Automate the process of log analysis and intent classification.
            \item Provide security professionals with insights into attack strategies.
        \end{itemize}
        
        The significance of this research lies in its potential to enhance cybersecurity threat detection and response capabilities by transforming complex, unstructured log data into actionable intelligence.
