% 05-unsupervised-learning-clustering.tex

% Unsupervised Learning – Clustering
% 5.1. Introduction: Provides an overview of the clustering task and its objectives.
% 5.2. Determine the Number of Clusters: Uses methods like the elbow method or silhouette analysis to determine the number of clusters.
% 5.3. Hyperparameter Tuning: Tunes other hyperparameters, if any.
% 5.4. Cluster Visualization: Visualizes the clusters through t-SNE.
% 5.5. Cluster Analysis: Analyzes the characteristics of each cluster.
% 5.6. Intent Homogeneity: Assesses if clusters reflect intent division.
% 5.7. Specific Attack Categories: Associates clusters with specific attack categories.

% Section Title
\section{UNSUPERVISED LEARNING - CLUSTERING}

    % Main Content

    \subsection{Introduction}
    
        Unsupervised learning, a powerful branch of machine learning, was applied in this project to gain insights from SSH attack data. The primary focus was on leveraging clustering methods to group similar attack sessions based on their intrinsic patterns and characteristics. By analyzing these groups, the study aimed to uncover hidden relationships and categorize different attack intents and behaviors without relying on predefined labels. 
    
    \subsection{Data Preparation}
    
        The dataset chosen was the one generated through the TF-IDF vectorization technique. This was made because it was essential to start with a dataset that represented in the best way the frequency and the importance of words, making each word as a dimension of our vector.

    \subsection{Clustering Methods}
    
        Clustering techniques were employed to uncover natural groupings within the dataset, providing insights into SSH attack patterns. The following methods were used:
        
        \begin{itemize}
        
            \item \textbf{K-Means Clustering}: The algorithm iteratively assigns each data point to the nearest cluster centroid and updates the centroids until convergence. The Elbow Method was applied to determine the optimal number of clusters by examining the total within-cluster sum of squares (inertia). Silhouette scores were also calculated to evaluate the cohesion and separation of clusters, ensuring the clustering results were meaningful and well-separated.
            
            \item \textbf{Gaussian Mixture Model (GMM)}: Unlike K-Means, GMM considers the probability of each data point belonging to a cluster, providing a more flexible and nuanced clustering approach. The optimal number of clusters was determined using a combination of log-likelihood scores, which measure how well the model fits the data, and silhouette analysis to validate cluster quality. This dual approach ensured the GMM provided reliable and interpretable clustering results.
            
        \end{itemize}
        
    \subsection{Clustering Evaluation Techniques}
    
        \subsubsection{K-Means Clustering \\}
            
            The evaluation of the K-Means clustering results is based on the Elbow Method and the Silhouette Score, which provide insights into the optimal number of clusters for the dataset. The Elbow Method graph shows a steep decline in clustering error between 3 and 6 clusters, followed by a more gradual decrease as the number of clusters increases. The point of inflection, or "elbow," appears around 6 clusters, suggesting that adding more clusters beyond this point results in diminishing improvements in minimizing intra-cluster variance. The Silhouette Score graph exhibits a rapid increase up to 5 clusters, reaching a stable high value of approximately 0.95. A drop is observed around 8 clusters, after which the score gradually increases again, peaking beyond 12 clusters. This pattern indicates that a smaller number of clusters (around 5-6) achieves a strong balance of cohesion and separation, while additional clusters beyond 12 continue to refine the structure with marginal improvements. Considering both metrics, the optimal number of clusters for K-Means is likely between 5 and 6, ensuring a trade-off between clustering accuracy and computational efficiency.
        
        \subsubsection{Gaussian Mixture Model (GMM) \\}

            The evaluation of the GMM clustering results is based on the Silhouette Score and the Log-Likelihood Score, both of which provide insights into the quality of the clustering structure.

            The Silhouette Score graph shows a sharp increase up to 5 clusters, reaching a stable high value around 0.95. A slight drop is observed at 8 clusters, followed by a steady increase, with the highest scores occurring beyond 12 clusters. This suggests that increasing the number of clusters generally improves separation and cohesion, though the optimal balance appears to be around 6 clusters, where the highest stable performance is first achieved.

            The Log-Likelihood Score graph indicates a rapid increase from 3 to 5 clusters, after which the improvements become more gradual. Beyond 12 clusters, the score stabilizes, indicating diminishing returns in model fitting. This suggests that while increasing the number of clusters provides better representation of the data, the most significant improvements occur within the first few increments.

            Considering both metrics, an optimal cluster configuration is likely between 6 and 8 clusters, balancing cluster separation, model likelihood, and computational efficiency..

        % \vspace{-0.3cm}
        
        \begin{figure}[h]
            \centering
            \begin{minipage}[c]{0.47\textwidth}
                \centering
                \includegraphics[width=\textwidth]{../figures/plots/section3/k-means_clustering_error.png}
                \caption{K-means Elbow Method}
                \label{fig:tsne_kmeans}
            \end{minipage}
            \hfill
            \begin{minipage}[c]{0.47\textwidth}
                \centering
                \includegraphics[width=\textwidth]{../figures/plots/section3/k-means_silohuette_score.png}
                \caption{K-means Silhouette Score}
                \label{fig:tsne_gmm}
            \end{minipage}
        \end{figure}
        
        % \vspace{-0.5cm}
        
        \begin{figure}[h]
            \centering
            \begin{minipage}[c]{0.47\textwidth}
                \centering
                \includegraphics[width=\textwidth]{../figures/plots/section3/gmm_total_log-likelihood_score.png}
                \caption{GMM Log-Likelihood Score}
                \label{fig:tsne_kmeans}
            \end{minipage}
            \hfill
            \begin{minipage}[c]{0.47\textwidth}
                \centering
                \includegraphics[width=\textwidth]{../figures/plots/section3/gmm_silohuette_score.png}
                \caption{GMM Silhouette Score}
                \label{fig:tsne_gmm}
            \end{minipage}
        \end{figure}

    \subsection{Hyperparameter Tuning}
    
        Hyperparameter tuning was conducted with Grid Search method, to optimize the performance of both clustering methods. For K-Means, parameters such as the initialization method (\texttt{k-means++} and \texttt{random}), the number of initializations, and the maximum number of iterations were fine-tuned using a grid search approach. Similarly, GMM parameters including the initialization method (\texttt{kmeans}), covariance type (\texttt{full} and \texttt{spherical}), and tolerance were optimized. These steps ensured the models were tailored to the dataset, resulting in better clustering outcomes.

    \subsection{Clusters Visualization}
    
        To visualize the clustering results, t-SNE dimensionality reduction was applied. Its functionalities makes it an excellent choice for visualizing clusters in datasets where direct interpretation is difficult due to high dimensionality. By projecting the data into a two-dimensional space, t-SNE enables us to identify patterns and groupings that may not be evident in the original feature space. The two-dimensional plots provided a clear representation of the clusters formed by both K-Means and GMM, highlighting their separability and internal consistency.
        
        \begin{figure}[h]
            \centering
            \begin{minipage}[c]{0.47\textwidth}
                \centering
                \includegraphics[width=\textwidth]{../figures/plots/section3/tsne_kmeans_clusters.png}
                \caption{t-SNE Visualization of K-Means Clusters}
                \label{fig:tsne_kmeans}
            \end{minipage}
            \hfill
            \begin{minipage}[c]{0.47\textwidth}
                \centering
                \includegraphics[width=\textwidth]{../figures/plots/section3/tsne_gmm_clusters.png}
                \caption{t-SNE Visualization of GMM Clusters}
                \label{fig:tsne_gmm}
            \end{minipage}
        \end{figure}

        \subsubsection{K-Means Visualization \\}

            The t-SNE visualization provides a two-dimensional representation of the 10 clusters identified by the K-Means algorithm. This graph highlights the spatial distribution and separability of the clusters in the dataset.

            Distinct cluster groupings are visible, with some clusters (e.g., the orange and purple clusters) forming highly compact and dense regions. This indicates strong intra-cluster similarity and effective separation from other clusters. Conversely, a few clusters (e.g., the red and yellow clusters) are more dispersed, suggesting potential overlap or variability in their features.

            The visualization demonstrates that K-Means effectively partitions the dataset into meaningful groups, with well-defined boundaries for most clusters. However, the presence of dispersed clusters may reflect the complexity of certain patterns in the dataset, indicating that some attack behaviors share overlapping features. Overall, the t-SNE plot validates the clustering results by showing clear differentiation among the majority of the clusters.
      
        \subsubsection{GMM Visualization \\}

            The clusters exhibit distinct groupings, with some (e.g., the red and blue clusters) forming tightly packed regions that indicate strong intra-cluster similarity and minimal overlap. However, other clusters (e.g., the orange cluster) are more dispersed, reflecting the probabilistic nature of GMM, which allows for overlapping data points and accounts for uncertainties in cluster assignment.

            Compared to the K-Means visualization, the GMM-based clusters appear more balanced in size and density. This suggests that GMM is effectively capturing the natural variability and overlapping characteristics within the dataset. Overall, the t-SNE plot validates the clustering results, showing that GMM provides a nuanced partitioning of the dataset while accommodating the complexities inherent in the data.

    \subsection{Clusters Analysis}

        \subsubsection{Word Cloud Representation \\}

            Word clouds were generated for each cluster to highlight the most significant terms. These visualizations provided an intuitive understanding of the key features within each cluster by emphasizing frequently occurring terms. The approach helped in identifying the distinguishing characteristics of each cluster, offering insights into the behavioral patterns and intents associated with different attack sessions.

            The word clouds revealed dominant keywords for specific clusters, such as commands, parameters, or phrases frequently used in SSH attacks. This information serves as a valuable reference for understanding the nature of the clustered attack sessions, aiding in further analysis and interpretation of the underlying data.

            \begin{figure}[H]
                \centering
                \includegraphics[width=0.9\textwidth]{../figures/plots/section3/circular_wordclouds.png}
                \caption{Word Clouds for Each Cluster}
                \label{fig:word_clouds}
            \end{figure}
            
            \subsubsection{Community Detection \\}
            
                Graph-based community detection was performed within selected clusters to identify subgroups. This approach involves constructing a graph where nodes represent features or sessions, and edges indicate relationships or co-occurrences within the data. The greedy\_modularity\_communities method was applied to detect communities, maximizing modularity to ensure well-defined groups.

                The analysis revealed meaningful relationships and substructures within the clusters, helping to further refine the understanding of the dataset's internal patterns. These subgroups highlight finer-grained distinctions within the clusters, such as frequently occurring command sequences or behavioral traits common to specific attack types. Visualizations of these detected communities provide insights into the interconnectedness of the data and potential hierarchical structures in SSH attack patterns.

                \begin{figure}[h]
                    \centering
                    \begin{minipage}[c]{0.47\textwidth}
                        \centering
                        \includegraphics[width=0.8\textwidth]{../figures/plots/section3/k-means_graph_visualization_of_cluster_0_with_communities.png}
                        \caption{Community Detection in Cluster 0 (K-Means).}
                        \label{fig:kmeans_graph}
                    \end{minipage}
                    \hfill
                    \begin{minipage}[c]{0.47\textwidth}
                        \centering
                        \includegraphics[width=0.8\textwidth]{../figures/plots/section3/gmm_graph_visualization_of_cluster_0_with_communities.png}
                        \caption{Community Detection in Cluster 0 (GMM).}
                        \label{fig:gmm_graph}
                    \end{minipage}
                \end{figure}
            
                These visualizations demonstrate that distinct subgroups exist within Cluster 0. For K-Means, the graph reveals tight communities (e.g., green or pink) that suggest strong intra-community similarities, whereas more dispersed connections (e.g., cyan or orange) indicate weaker relationships. In contrast, the GMM graph shows balanced community sizes, reflecting GMM's probabilistic approach, which accounts for overlapping data points. 

                Overall, these detected communities provide finer-grained categorizations of attack patterns, enabling a deeper understanding of SSH attack behaviors beyond standard clustering methods.

    \subsection{Conclusion}
    
        This analysis provided valuable insights into the patterns of SSH attacks through clustering. The K-Means and GMM algorithms both effectively identified meaningful clusters, as supported by validation metrics and visualizations. Hyperparameter tuning further enhanced the performance of both models. The results demonstrate the potential of unsupervised learning in uncovering hidden patterns in complex datasets, providing a foundation for future applications such as anomaly detection and improved cybersecurity strategies.
