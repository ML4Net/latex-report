% 02-background.tex

% Background: Explanation of the context, including the importance of security logs, the MITRE ATT&CK framework, and the dataset used.

% Section Title
\section{BACKGROUND}

    % Main Content

    \subsection{Security Logs and Attack Analysis}
    
        % Security logs represent a critical source of information for understanding system vulnerabilities and potential cyber attacks. These logs capture detailed records of system events, network interactions, and user activities, providing valuable insights into potential security breaches.

        In the context of SSH shell attacks, logs document the sequence of commands executed during a malicious session, enabling security researchers to analyze attacker behaviors, techniques, and potential system impacts. However, the manual analysis of these logs is challenging due to their volume, complexity, and often non-standard formatting.

    \subsection{MITRE ATT\&CK Framework}
    
        The MITRE ATT\&CK (Adversarial Tactics, Techniques, and Common Knowledge) framework provides a comprehensive knowledge base of adversary tactics and techniques observed in real-world cyber attacks. This framework serves as a standardized methodology for understanding and categorizing attack strategies.

        \noindent For our research, we focus on seven key intents derived from the MITRE ATT\&CK framework:

        \begin{itemize}
            \item \textbf{Persistence:} Techniques used by adversaries to maintain system access across restarts or credential changes.
            \item \textbf{Discovery:} Methods for gathering information about the target system and network environment.
            \item \textbf{Defense Evasion:} Strategies to avoid detection by security mechanisms.
            \item \textbf{Execution:} Techniques for running malicious code on target systems.
            \item \textbf{Impact:} Actions aimed at manipulating, interrupting, or destroying systems and data.
            \item \textbf{Other:} Less common tactics including Reconnaissance, Resource Development, Initial Access, etc.
            \item \textbf{Harmless:} Non-malicious code or actions.
        \end{itemize}

    \subsection{Research Approach}
    
        Our research employs a multi-faceted approach to SSH shell attack log analysis:

        \begin{itemize}
            \item Explore and preprocess a large dataset of Unix shell attack sessions.
            \item Apply supervised learning techniques to classify attack tactics based on session characteristics.
            \item Utilize unsupervised learning methods to discover patterns and clusters in attack sessions.
            \item Investigate the potential of advanced language models in understanding and categorizing attack intents.
        \end{itemize}

        % By combining these techniques, we aim to develop a comprehensive framework for understanding and categorizing SSH shell attacks, ultimately contributing to improved cybersecurity threat detection and response strategies.
