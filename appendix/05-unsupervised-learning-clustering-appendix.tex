% 05-unsupervised-learning-clustering-appendix.tex

% Section Title
\section{UNSUPERVISED LEARNING - CLUSTERING}

    % Main Content
    
    % TODO: add subsections
    % TODO: add images

    For K-Means, the community detection process segmented Cluster 0 into distinct subgroups, each associated with different types of attack-related commands. The identified communities are:

        \begin{itemize}
        \item \textbf{Community 0:} Contains key system administration commands such as \texttt{var}, \texttt{wget}, and \texttt{x19}, indicating potential reconnaissance or system modification behavior.
        \item \textbf{Community 1:} Includes numeric sequences and encoded strings, suggesting automated attack scripts or encoded payloads.
        \item \textbf{Community 2:} Features authentication and privilege escalation commands like \texttt{admin}, \texttt{authorizedkeys}, and \texttt{bash}, often linked to unauthorized access attempts.
        \item \textbf{Community 3:} Focuses on system information gathering with commands like \texttt{cpuinfo}, \texttt{chmod}, and \texttt{cat}, which could indicate enumeration activities.
        \item \textbf{Community 4:} Contains commands such as \texttt{dota}, \texttt{done}, and \texttt{echo}, possibly linked to custom scripts or malware execution.
        \item \textbf{Community 5:} Features command-line utilities like \texttt{grep}, \texttt{enable}, and \texttt{head}, potentially associated with reconnaissance.
        \item \textbf{Community 6:} Includes directory and file operations such as \texttt{ls}, \texttt{mkdir}, and \texttt{mem}, suggesting file system exploration.
        \item \textbf{Community 7:} Focuses on credential-related commands like \texttt{passwd}, \texttt{password}, and \texttt{mounts}, linked to credential theft or privilege escalation.
        \item \textbf{Community 8:} Contains data transfer and file manipulation commands like \texttt{rsync}, \texttt{rm}, and \texttt{root}, indicating data exfiltration or malware deployment.
        \item \textbf{Community 9:} Features execution-related commands such as \texttt{ssh}, \texttt{system}, and \texttt{tar}, likely related to persistence or remote command execution.
        \item \textbf{Community 10:} Includes Unix-specific commands like \texttt{uname}, \texttt{tmp}, and \texttt{tftp}, which may indicate environment probing or lateral movement techniques.
        \end{itemize}

        For the Gaussian Mixture Model (GMM), community detection identified similar subgroups, but with a more flexible clustering approach, allowing overlap between different command sets. The detected communities are:

        \begin{itemize}
        \item \textbf{Community 0:} Shows a focus on system administration commands like \texttt{wget}, \texttt{x19}, and \texttt{var}, which are frequently seen in exploit scripts.
        \item \textbf{Community 1:} Contains encoded strings and random character sequences, which might indicate obfuscated payloads.
        \item \textbf{Community 2:} Includes \texttt{admin}, \texttt{bash}, and \texttt{authorizedkeys}, aligning with privilege escalation or credential manipulation.
        \item \textbf{Community 3:} Displays enumeration commands like \texttt{cpuinfo}, \texttt{chmod}, and \texttt{cat}, highlighting system profiling.
        \item \textbf{Community 4:} Consists of custom script keywords and execution utilities such as \texttt{dota}, \texttt{done}, and \texttt{echo}.
        \item \textbf{Community 5:} Groups command-line search utilities like \texttt{grep}, \texttt{enable}, and \texttt{head}, useful for filtering and reconnaissance.
        \item \textbf{Community 6:} Represents file system operations with \texttt{ls}, \texttt{mkdir}, and \texttt{mem}, suggesting persistence mechanisms.
        \item \textbf{Community 7:} Includes credential-related commands like \texttt{passwd}, \texttt{password}, and \texttt{mounts}, linked to unauthorized access.
        \item \textbf{Community 8:} Contains commands such as \texttt{rsync}, \texttt{rm}, and \texttt{root}, commonly used in data exfiltration scenarios.
        \item \textbf{Community 9:} Features execution and session management commands such as \texttt{ssh}, \texttt{system}, and \texttt{tar}, indicating persistence techniques.
        \item \textbf{Community 10:} Includes Unix environment commands like \texttt{uname}, \texttt{tmp}, and \texttt{tftp}, which might be used for reconnaissance or system information gathering.
        \end{itemize}
    
    \begin{lstlisting}[caption={Cluster 0 - K-Means Community Analysis}, label={lst:cluster-0-k-means-community-analysis}]
            Cluster 0 - K-Means Community Analysis:

            Community 0: ['var', 'wc', 'while', 'wget', 'x13', 'which', 'x17', 'x19', 'xf']

            Community 1: ['8m', '20m', '15s', '0kx34uax1rv', '172', '75gvomnx9euwonvnoaje0 qxxziig9elbhp glmuakb5bgtfbrkjaw9u9fstdengvs8hx1knfs4mjux0hjok8rvcempecjdy symb66nylakgwcee6weqhmd1mupghwgq0hwcwsqk13ycgpk5w6hyp5zykfnvlc8hgmd4wwu97k 6pftgtubjk14ujvcd 9iukqttwyyjiiu5pmuux5bsz0r4wfwdie6i6rblaspkgaysvkprkorw','192', '3s']

            Community 2: ['admin', 'bs', '9p7vd0epz3tz', 'bin', 'aaaab3nzac1yc2eaaaabjqaaa qeardp4cun2lhr4kuhbge7vvacwdli2a8dbnrtorbmz15o73fcbox8nvbut0buanuv9tj2',   'awk', 'authorizedkeys', 'bash']

            Community 3: ['cpuinfo', 'chpasswd', 'chmod', 'cat', 'busybox', 'cd', 'count', 'cp']

            Community 4: ['dota', 'done', 'do', 'dota3', 'dev', 'dd', 'echo', 'crontab']

            Community 5: ['enable', 'go', 'free', 'exit', 'grep', 'gz', 'if', 'head']

            Community 6: ['ls', 'initall', 'mdrfckr', 'lscpu', 'model', 'mkdir', 'lh',     'mem']

            Community 7: ['mv', 'passwd', 'null', 'name', 'mounts', 'password', 'nohup',   'mountfs']

            Community 8: ['proc', 'rsync', 'root', 'rsa', 'rm', 'rf', 'read', 'print']

            Community 9: ['shm', 'ssh', 'system', 'tar', 'systemcache', 'sleep', 'sh',     'shell']

            Community 10: ['tftp', 'tsm', 'unix', 'up', 'top', 'txt', 'tmp', 'uname']
            
    \end{lstlisting}
    
    \begin{lstlisting}[caption={Cluster 0 - GMM Community Analysis}, label={lst:cluster-0-gmm-community-analysis}]
            Cluster 0 - GMM Community Analysis:
            
            Community 0: ['var', 'wc', 'while', 'wget', 'x13', 'which', 'x17', 'x19', 'xf']

            Community 1: ['8m', '20m', '15s', '0kx34uax1rv', '172', '75gvomnx9euwonvnoaje0 qxxziig9elbhpglmuakb5bgtfbrkjaw9u9fstdengvs8hx1knfs4mjux0hjok8rvcempecjdys ymb66nylakgwcee6weqhmd1mupghwgq0hwcwsqk13ycgpk5w6hyp5zykfnvlc8hgmd4wwu97k6 pftgtubjk14ujvcd9iukqttwyyjiiu5pmuux5bsz0r4wfwdie6i6rblaspkgaysvkprkorw',  .'192', '3s']

            Community 2: ['admin', 'bs', '9p7vd0epz3tz', 'bin', 'aaaab3nzac1yc2eaaaabjqaa aqeardp4cun2lhr4kuhbge7vvacwdli2a8dbnrtorbmz15o73fcbox8nvbut0buanuv9tj2',  'awk', 'authorizedkeys', 'bash']

            Community 3: ['cpuinfo', 'chpasswd', 'chmod', 'cat', 'busybox', 'cd', 'count', 'cp']

            Community 4: ['dota', 'done', 'do', 'dota3', 'dev', 'dd', 'echo', 'crontab']

            Community 5: ['enable', 'go', 'free', 'exit', 'grep', 'gz', 'if', 'head']

            Community 6: ['ls', 'initall', 'mdrfckr', 'lscpu', 'model', 'mkdir', 'lh',     'mem']

            Community 7: ['mv', 'passwd', 'null', 'name', 'mounts', 'password', 'nohup',   'mountfs']

            Community 8: ['proc', 'rsync', 'root', 'rsa', 'rm', 'rf', 'read', 'print']

            Community 9: ['shm', 'ssh', 'system', 'tar', 'systemcache', 'sleep', 'sh',    'shell']

            Community 10: ['tftp', 'tsm', 'unix', 'up', 'top', 'txt', 'tmp', 'uname']
    
    \end{lstlisting}
    
    \clearpage
    
    \clearpage
    